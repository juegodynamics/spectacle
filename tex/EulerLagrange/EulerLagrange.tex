\documentclass{article}
\usepackage[margin=1in]{geometry} 
\usepackage{amsmath, amssymb, graphics, setspace, tcolorbox, fancyhdr, accents, cancel}
\pagestyle{fancy}
\setlength{\headheight}{20pt}

\newcommand{\DD}[2]{\frac{d #1}{d #2}}
\newcommand{\At}[1]{\Bigr|_{#1}}
\newcommand{\PD}[2]{\frac{\partial #1}{\partial #2}}

\begin{document}
\lhead{Jacovie Rodriguez}
\rhead{Explorations}

A Lagrangian is dependent on a given function $f$:
\begin{equation}
    L[f] = L \left( x, f(x), f'(x) \right)
\end{equation}

The Action is defined as a path integral for the Lagrangian:
\begin{equation}
    A[f] = \int_a^b L[f] dx
\end{equation}

Suppose $f(x)$ extremizes $A$. Then let's define $\zeta$ as small arbitrary perturbation of $f$:
\begin{equation}
    \label{eqn:perturbation}
    \zeta(x) = f(x) + \epsilon \eta(x)
\end{equation}
\begin{itemize}
    \setlength{\itemindent}{2em}
    \item The $\eta(x)$ is the difference between $f$ and its perturbation $\zeta$. Since $\zeta$ must start on $a$ and end on $b$ just as $f$, this requires that $\eta(a) = \eta(b) = 0$.
    \item The $\epsilon$ here is a small constant. By definition, $\zeta \to f$ as $\epsilon \to 0$.
\end{itemize}

If we analyze the perturbed Action
\begin{equation}
    A[\zeta] = \int_a^b L \left( \zeta \right) dx
\end{equation}

\hspace{\parindent} then we see as $\epsilon \to 0$, $A[\zeta] \to A[f]$.
\\

Since $A[f]$ is an extremum, then it must be the case that:
\begin{equation}
    \label{eqn:actionlimit}
    \lim_{\epsilon \to 0} \DD{A[\zeta]}{\epsilon} = 0
\end{equation}

Let's see how we can evaluate the total derivative via chain rule:
\begin{align}
    \DD{A[\zeta]}{\epsilon}
     & = \DD{}{\epsilon} \int_a^b L [\zeta] dx \\
     & = \int_a^b \DD{L}{\epsilon} dx          \\
     & = \int_a^b \left[
        \DD{x}{\epsilon} \PD{L}{x}
        + \DD{\zeta}{\epsilon} \PD{L}{\zeta}
        + \DD{\zeta'}{\epsilon} \PD{L}{\zeta'}
        \right] dx
\end{align}

We can evaluate each total derivative:

\begin{center}
    \begin{tabular}{ c|c|c }
        $\displaystyle \DD{x}{\epsilon}  = 0$
         & $\displaystyle \DD{\zeta}{\epsilon}  = \eta(x)$
         & $\displaystyle \DD{\zeta'}{\epsilon} = \eta'(x)$   \\
         &                                                  & \\
        since $x$ doesn't depend on $\epsilon$
         & since $\zeta(x) = f(x) + \epsilon \eta(x)$
         & since $\zeta'(x) = f'(x) + \epsilon \eta'(x)$      \\
    \end{tabular}
\end{center}

Altogether then,
\begin{align}
    \DD{A[\zeta]}{\epsilon}
     & = \int_a^b
    \left[
        \eta(x) \PD{L}{\zeta}
        + \eta'(x) \PD{L}{\zeta'}
    \right] dx    \\
    \lim_{\epsilon \to 0}
    \DD{A[\zeta]}{\epsilon}
     & = \int_a^b
    \left[
        \eta(x) \PD{L}{f}
        + \eta'(x) \PD{L}{f'}
        \right] dx
    = 0
\end{align}

\hspace{\parindent} where the last line goes to zero following from (\ref{eqn:actionlimit}).
\\

Now, we have information about $\eta$ and its axes, but not $\eta'$.


\end{document}